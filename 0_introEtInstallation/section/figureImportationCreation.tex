\section{Figures, importation et création}

\begin{frame}{Notes}
    \begin{alertblock}{Package}
        Pour importer des images dans votre document, il faut utiliser le package \texttt{graphicx}.
        Sans lui vous pourrez importer des images dans le document \texttt{.tex} mais elles n'apparaîtront pas dans le \texttt{.pdf}.
    \end{alertblock}
\end{frame}

\begin{frame}[containsverbatim]
    \frametitle{Importation d'une figure}
    \begin{exampleblock}{Contenu minimal recommandé}
        \begin{figure}
            \centering
            \includegraphics[scale=0.2]{LaTeX_logo.png}
            \caption{Logo de \LaTeX}
            \label{fig:logoLatex}
        \end{figure}
    \end{exampleblock}
    \begin{lstlisting}[language=TeX]
        \begin{figure}
            \centering
            \includegraphics[scale=0.2]{LaTeX_logo.png}
            \caption{Logo de \LaTeX} %la table des figs.
            \label{fig:logoLatex} %utile pour les cf.
        \end{figure}
    \end{lstlisting}
\end{frame}

\begin{frame}[containsverbatim]
    \frametitle{Divers paramètres}
    \begin{exampleblock}{Contenu minimal recommandé}
        \begin{figure}
            \centering
            \includegraphics[width=1cm, angle=35]{LaTeX_logo.png}
            \caption{Logo de \LaTeX}
        \end{figure}
    \end{exampleblock}
    \begin{lstlisting}[language=TeX]
        \begin{figure}
            \centering
            \includegraphics[width=1cm, angle=160]{LaTeX_logo.png}
            \caption{Logo de \LaTeX}
            \label{fig:logoLatex}
        \end{figure}
    \end{lstlisting}
\end{frame}


\begin{frame}[containsverbatim]
    \frametitle{Dessin vectoriel}
    \begin{alertblock}{Packages possibles}
        \begin{itemize}[label=$\triangleright$]
            \item \texttt{tikz}
            \item \texttt{pstricks}
            \item \texttt{pst-node}
        \end{itemize}
    \end{alertblock}
    \begin{alertblock}{Alternatives}
        Importer directement des fichiers vectoriels avec les extensions adéquats (\texttt{*.eps}, \texttt{*.svg}, \dots) \\
        ou bien \\
        Le site internet \href{https://www.mathcha.io/}{https://www.mathcha.io/} qui permet de créer le dessin et générer le code TeX
    \end{alertblock}
\end{frame}