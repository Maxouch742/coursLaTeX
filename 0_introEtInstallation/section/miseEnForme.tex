\section{Mise en forme du texte}

\begin{frame}[containsverbatim]
    \frametitle{Mise en forme du texte}
    \begin{table}
        \centering
        \begin{tabular}{|c||c|c|}
            \hline
            Gras     & \textbf{géodésie}    & \verb|\textbf{géodésie}| \\
            \hline
            Italique & \textit{géodésie}    & \verb|\textit{géodésie}| \\
            \hline
            Souligné & \underline{géodésie} & \verb|\underline{géodésie}| \\
            \hline
            Télétype & \texttt{géodésie} & \verb|\texttt{géodésie}| \\
            \hline
            Sans-sérif & \textsf{géodésie} & \verb|\textsf{géodésie}| \\
            \hline
            Incliné & \textsl{géodésie} & \verb|\textsl{géodésie}| \\
            \hline
            Petites majuscules & \textsc{géodésie} & \verb|\textsc{géodésie}| \\
            \hline
        \end{tabular}
    \end{table}
\end{frame}

\begin{frame}[containsverbatim]
    \frametitle{Taille de police}
    \begin{table}
        \centering
        \begin{tabular}{|c|c|}
            \hline
            \tiny{géodésie} & \verb|\tiny{géodésie}| \\
            \hline
            \scriptsize{géodésie} & \verb|\scriptsize{géodésie}| \\
            \hline
            \footnotesize{géodésie} & \verb|\footnotesize{géodésie}| \\
            \hline
            \small{géodésie} & \verb|\small{géodésie}| \\
            \hline
            \normalsize{géodésie} & \verb|\normalsize{géodésie}| \\
            \hline
            \large{géodésie} & \verb|\large{géodésie}| \\
            \hline
            \Large{géodésie} & \verb|\Large{géodésie}| \\
            \hline
            \LARGE{géodésie} & \verb|\LARGE{géodésie}| \\
            \hline
            \huge{géodésie} & \verb|\huge{géodésie}| \\
            \hline
            \Huge{géodésie} & \verb|\Huge{géodésie}| \\
            \hline
        \end{tabular}
    \end{table}
\end{frame}

\begin{frame}[containsverbatim]
    \frametitle{Modification de la taille de la police}
    \begin{lstlisting}[language=TeX, caption=Différentes manières]
    a {\Large texte} b
    a \Large texte \normalsize b
    a \begin{Large}texte\end{Large} b
    \end{lstlisting}
    \begin{exampleblock}{Résultat}
        a {\Large géodésie} b
    \end{exampleblock}
    \begin{alertblock}{Attention}
        Ce sont des tailles relatives par rapport à la taille de police définie dans la première ligne du \texttt{main.tex}
    \end{alertblock}
\end{frame}