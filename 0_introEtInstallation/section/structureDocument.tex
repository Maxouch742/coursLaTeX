\section{Structure (paragraphes, \dots)}

\begin{frame}[containsverbatim]
    \frametitle{Chapitres, paragraphes, \dots}
    \begin{table}
        \centering
        \begin{tabular}{|l|l|}
            \hline
            Partie            & \verb|\part{titre}| \\
            \hline
            Chapitre          & \verb|\chapter{titre}| \\
            \hline
            Section           & \verb|\section{titre}| \\
            \hline
            Sous-section      & \verb|\subsection{titre}| \\
            \hline
            Sous-sous-section & \verb|\subsubsection{titre}| \\
            \hline
            Paragraphe        & \verb|\paragraph{titre}| \\
            \hline
        \end{tabular}
    \end{table}
    
\end{frame}

\begin{frame}{Exemples -- Chapitres, paragraphes, \dots}
    \encours
\end{frame}

\begin{frame}[containsverbatim]
    \frametitle{Absence de numérotation}
    Il est parfois utile de ne pas numéroter un chapitre ou une section (par exemple l'introduction).
    Il suffit pour cela d'ajouter un astérix (*) après le chapitre. \\
    Exemple : \verb|\chapter*{titre}|
    \bigskip
    \begin{exampleblock}{Ajouter le titre non numéroté au sommaire}
        \encours
    \end{exampleblock}
\end{frame}