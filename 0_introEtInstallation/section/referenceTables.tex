\section{Références et tables}

\begin{frame}[containsverbatim]
    \frametitle{Création de référence}
    \begin{alertblock}{Explication}
        Il est tout à possible de créer une \og ancre \fg{} à un paragraphe, une équation ou une figure ou tout autre élément de votre choix. \\
        La commande à utiliser est : \verb|\label{}|
    \end{alertblock}
    \begin{lstlisting}[language=TeX, caption=Exemple]
    \begin{equation}\label{eq:relativite}
        e = m \cdot c^2
    \end{equation}
    \end{lstlisting}
    \begin{equation}\label{eq:relativite}
        e = m \cdot c^2
    \end{equation}
\end{frame}

\begin{frame}[containsverbatim]
    \frametitle{Référence à une ancre}
    \begin{alertblock}{Explication}
        Faire référence à une ancre est une méthode générique. \\
        Pour cela, la commande à utiliser est : \verb|\ref{}|.
    \end{alertblock}
    \begin{lstlisting}[language=TeX, caption=Exemple]
    L'equation \ref{eq:relativite} est la formule de la
    relavitive generale.
    \end{lstlisting}
    L'équation \ref{eq:relativite} est la formule de la relativité générale.
\end{frame}

\begin{frame}[containsverbatim]
    \frametitle{Référence à une page}
    \begin{alertblock}{Explication}
        On veut parfois savoir à quelle page se situe l'objet dont on fait la référence. \\
        Pour cela, on peut utiliser la commande : \verb|\pageref{}|
    \end{alertblock}
    \begin{lstlisting}[language=TeX, caption=Exemple]
    L'equation \ref{eq:relativite}, page
    \pageref{eq:relativite} est la formule de la 
    relavitive generale.
    \end{lstlisting}
    L'équation \ref{eq:relativite}, page \pageref{eq:relativite} est la formule de la relativité générale.
    \end{frame}

\begin{frame}[containsverbatim]
    \frametitle{Référence à l'objet courant}
    \begin{alertblock}{Explication}
        Il est courant de faire référence parfois au chapitre, ou section dans lequel on se trouve. \\
        Pour cela on utilise des commandes : \texttt{the}-commandes.
    \end{alertblock}
    \begin{lstlisting}[language=TeX, caption=Exemple]
    Nous sommes actuellement dans la section \thesection.
    \end{lstlisting}
    Nous sommes actuellement dans la section \thesection.
\end{frame}

\begin{frame}[containsverbatim]
    \frametitle{Création d'une table (ou liste)}
    \begin{alertblock}{Explication}
        Grâce à la structure du document (chapitres, images, tableaux, \dots), il est possible d'obtenir la table des matières et les tables d'images aisément. \\
        De plus, en compilant le fichier, les tables se mettent à jour automatiquement ; ainsi que les références etc.
    \end{alertblock}
    \footnotesize
    \begin{table}
        \centering
        \begin{tabular}{|l|l|}
            \hline
            \verb|\tableofcontents| & Table des matières  \\
            \verb|\listoffigures|   & Liste des figures   \\
            \verb|\listoftables|    & Listes des tableaux \\
            \hline
        \end{tabular}
    \end{table}
    \begin{exampleblock}{Nota}
        Il n'est pas possible de créer une table des équations comme on fait pour les images. \\
        Il existe d'autres alternatives en créant une macro qui légende les équations ou alors de créer un index avec des mots particuliers.
    \end{exampleblock}
\end{frame}
\normalsize
