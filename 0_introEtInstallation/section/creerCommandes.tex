\section{Créer ses propres commandes}

\begin{frame}[containsverbatim]
    \frametitle{Macro-commande sans option}
    \begin{lstlisting}[language=TeX, caption=Forme de la commande]
        \newcommand{nom}{description}
    \end{lstlisting}
    \begin{lstlisting}[language=TeX, caption=Exemples]
    \newcommand{\be}{\begin{equation}}
    \newcommand{\ee}{\end{equation}}
    \newcommand{\ptF}{{\color{red} \textbf{Pt Fixe}}}
    \newcommand{\ptN}{{\color{green} \textbf{Pt Nouv.}}}
    \end{lstlisting}
\end{frame}

\begin{frame}[containsverbatim]
    \frametitle{Macro-commande avec options}
    \begin{lstlisting}[language=TeX, caption=Forme de la commande]
        \newcommand{nom}[nombre d'arguments]{description}
    \end{lstlisting}
    \begin{lstlisting}[language=TeX, caption=Exemples]
        \newcommand{\vvec}[1]{\overirghtarrow{#1}}
        \newcommand{\dis}[2]{d_{#1-#2}}
    \end{lstlisting}
\end{frame}
