
\section{Installation}
\frame{
    {Solutions possibles}
    Plusieurs solutions logicielles existent pour produire un document PDF à partir d'un document \LaTeX :
    \begin{itemize}
        \item \texttt{Overleaf} : éditeur de fichier \texttt{.tex} en ligne (site internet)
        \item \texttt{Tex Live} : environnement \LaTeX ; compatible avec le logiciel \texttt{Visual Studio Code}
        \item \texttt{MikTex} : environnement \LaTeX ; compatible entre autre avec le logiciel \texttt{TexMaker}
    \end{itemize}
}

\frame{
    {Overleaf}

    \begin{figure}
        \centering
        \includegraphics[height=1.5cm]{overleaf_logo.png}
    \end{figure}

    \vfill

    \begin{columns}
        \column{0.5\textwidth}
        \textbf{Avantages}
        \begin{itemize}
            \item Online
            \item Gratuit (licence étudiante)
            \item Exemple et templates disponibles
            \item Stockage online
            \item Possibilité de travailler à plusieurs
        \end{itemize}
        
        \column{0.5\textwidth}
        \textbf{Inconvénients}
        \begin{itemize}
            \item Online (besoin d'une connexion internet)
            \item Compile le fichier pour avoir un résultat à chaque fois
        \end{itemize}
    \end{columns}
    \begin{center}
        \href{https://www.overleaf.com}{\texttt{https://www.overleaf.com}}
    \end{center}
}

\frame{
    {TexLive \& VisualStudioCode}

    \begin{figure}
        \centering
        \subfigure{\includegraphics[height=1.5cm]{texlive_logo.jpg}} 
        \hspace{2cm}
        \subfigure{\includegraphics[height=1.5cm]{vsCode_logo.png}}
    \end{figure}

    \vfill

    \begin{columns}
        \column{0.5\textwidth}
        \textbf{Avantages}
        \begin{itemize}
            \item Desktop (toujours disponible)
            \item Compatible avec GitHub
            \item Configuration pour l'utilisateur
            \item Possibilité de connecter des scripts python pour générer les graphes (et mises à jour automatiques)
            \item Installation de tous les packages
        \end{itemize}
        
        \column{0.5\textwidth}
        \textbf{Inconvénients}
        \begin{itemize}
            \item Connaissance de base sur \LaTeX
            \item Peu user-friendly au premier abord
        \end{itemize}
    \end{columns}
}


\frame{
    {TexLive \& VisualStudioCode (Installation)}

    \begin{enumerate}
        \item Téléchargement :
        \begin{itemize}
            \item Tex Live : \href{https://www.tug.org/texlive/}{lien internet}
            \item VS Code : \href{https://code.visualstudio.com/download}{lien internet}
        \end{itemize}
        \item Configuration de VS Code :
        \begin{itemize}
            \item Télécharger l'extension \texttt{LaTeX Workshop} (James Yu)
            \item Suivre le tutoriel : \href{https://mathjiajia.github.io/vscode-and-latex/#step-3-install--configure-latex-workshop}{Jia Jia's Homepage}
        \end{itemize}
    \end{enumerate}
}

\frame{
    {MikTeX \& TexMaker}

    \begin{figure}
        \centering
        \subfigure{\includegraphics[height=1.5cm]{miktex_logo.png}} 
        \hspace{2cm}
        \subfigure{\includegraphics[height=1.5cm]{texmaker_logo.png}}
    \end{figure}

    \vfill

    \begin{columns}
        \column{0.5\textwidth}
        \textbf{Avantages}
        \begin{itemize}
            \item Desktop (toujours disponible)
            \item Utile pour les débutants (aides, commandes, ...)
        \end{itemize}
        
        \column{0.5\textwidth}
        \textbf{Inconvénients}
        \begin{itemize}
            \item Compilation parfois compliquée
            \item Erreurs mal détaillées
            \item Installation de certains packages
        \end{itemize}
    \end{columns}
}
